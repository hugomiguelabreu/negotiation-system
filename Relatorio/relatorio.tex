\documentclass[a4paper,12pt]{article}

%% Language and font encodings
\usepackage[portuguese]{babel}
\usepackage[utf8x]{inputenc}
\usepackage[T1]{fontenc}

%% Sets page size and margins
\usepackage[a4paper,top=3cm,bottom=2cm,left=3cm,right=3cm,marginparwidth=1.75cm]{geometry}

%% Useful packages
\usepackage{amsmath}
\usepackage{graphicx}
\usepackage[colorinlistoftodos]{todonotes}
\usepackage[colorlinks=true, allcolors=blue]{hyperref}
\usepackage{listings}
\usepackage{float}
\usepackage{hyphenat}
\usepackage{inconsolata}

\usepackage{color}
\usepackage{fancyhdr}
\pagestyle{fancy}
\lhead{FSD}
\rhead{Universidade do Minho - Departamento de Informática}

\setlength{\headheight}{13.6pt}

\definecolor{pblue}{rgb}{0.13,0.13,1}
\definecolor{pgreen}{rgb}{0,0.5,0}
\definecolor{pred}{rgb}{0.9,0,0}
\definecolor{pgrey}{rgb}{0.46,0.45,0.48}

\usepackage{listings}
\lstset{language=Java,
  showspaces=false,
  showtabs=false,
  breaklines=true,
  showstringspaces=false,
  breakatwhitespace=true,
  commentstyle=\color{pgreen},
  keywordstyle=\color{pblue},
  stringstyle=\color{pred},
  basicstyle=\ttfamily
}

\title{Sistema de negociação}
\author{Hugo Abreu | A76203 \and João Padrão | A76438\and João Reis | A75372
\\\\ \includegraphics[scale=0.25]{um_eeng} \\ \textbf{Universidade do Minho}}


\begin{document}

\maketitle


\section{Introdução}

\section{Front-end Server}

\section{Exchange Server}

\section{Directory Server}
Para projetar o servidor de directório, ou seja, o servidor que contém todos os meta-dados relativos ao sistema de negociação, portanto, esta entidade do sistema mantém dados como, que empreas existem, em que exchange são transacionadas, que exchanges existem e que empresas têm e, os diferentes preços diários das empresas em questão.
\par O servidor construído, tal como requirido pelo docente no enunciado, é um servidor REST, stateless, programado em Java com recuros a framework Dropwizard. Para ser possível garantir o acesso a todos os recursos que o serviço oferece, foi necessário definir \textit{end-points} para obter e/ou atualizar esses mesmos recursos. Os \textit{end-points} definidos para o servidor REST foram os seguintes:

\begin{itemize}
\item GET
  \begin{itemize}
    \item /companies
    \par Este \textit{end-point} devolve uma lista com todas as empresas existentes no sistema. A lista devolvida é uma lista de objetos e, portanto, contém todos os atributos intrínsecos a cada empresa na enumeração.
    \item /company/{id}
    \item /company/{id}/today
    \item /company/{id}/yesterday
    \item /exchange/{id}
    \item /exchange/{id}/companies
    \item /exchanges
  \end{itemize}
\item PUT
  \begin{itemize}
    \item /company/{id}/today
  \end{itemize}
\end{itemize}

\section{Client}


\end{document}